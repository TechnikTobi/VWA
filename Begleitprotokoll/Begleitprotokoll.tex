\documentclass[a4paper,12pt,ngerman,oneside]{scrreprt}

\usepackage[ngerman]{babel} 																		%Deutsche Einstellungen
\usepackage[utf8]{inputenc} 																		%UTF8
\usepackage[T1]{fontenc}																			%Schriftart für den Titel
\usepackage{csquotes}																				%Für Anführungszeichen
\usepackage[onehalfspacing]{setspace}																%1,5 Zeilenabstand
\usepackage{geometry}																				%Für die Bestimmung der Größe der Seiten
\usepackage[backend=bibtex, citestyle=authoryear]{biblatex}											%Fürs Literaturverzeichnis
\usepackage{graphicx}																				%Fürs Einbinden von Grafiken
\usepackage{wrapfig}																				%Fürs Verschieben von Grafiken
\usepackage{amsmath}																				%Für Vektoren
\usepackage{abstract}																				%Für das Abstract
\usepackage{url}																					%Für URLs im Literaturverzeichnis
\usepackage{etoolbox}	
\usepackage{chngcntr}
\usepackage{longtable}																				%Tabelle auf mehreren Seiten

\geometry{a4paper,left=20mm,right=20mm,top=10mm} 													%Format der Datei + Abstand zu den Rändern

\addtokomafont{chapter}{\rmfamily} 																	%Schriftart für Kapitelüberschriften
\addtokomafont{chapterentry}{\rmfamily}																%Schriftart für Kapiteleinstiege
\addtokomafont{section}{\rmfamily}																	%Schriftart für Unterkapitelüberschriften
\addtokomafont{subsection}{\rmfamily}																%Schriftart für Unterunterkapitelüberschriften
\addtokomafont{subsubsection}{\rmfamily} 															%Schriftart für Unterunterunterkapitelüberschriften

\setlength{\parindent}{0pt}																			%Nach Absatz nicht einrücken



\begin{document}
	\pagestyle{empty}
	\begin{center}	
		\vspace*{15mm}\huge\centering\textbf{Begleitprotokoll\break}
	\end{center}
	\vspace*{-5mm}
	Name der Schülerin/des Schülers: {Tobias Prisching}
	\newline Thema der Arbeit: {Künstliche Neuronale Netzwerke und ihr Verhalten beim  MNIST-Datensatz}
	\newline Name der Betreuungsperson: {Mag. Christoph Hödl}
	
	\vspace{1cm}
	\setlength\LTleft{0mm}
	\begin{longtable}{|p{21mm}|p{71mm}|p{66mm}|}
			
		%Datum & Vorgangsweise , ausgeführte Arbeiten, verwendete Hilfsmittel, aufgesuchte Bibliotheken, ... &  
		\hline
		Datum & Ausgeführte Arbeiten & Fortschritte/nächste Schritte \\
		\hline
		2017-11-02 & 1. Besprechung zum Thema mit betreuender Lehrperson & E-Mail mit Gliederung und Literatur \\\hline
		2017-11-17 & 1. E-Mail an die betreuende Lehrperson mit Gliederung und Literatur & \\\hline
		Herbst 2017 & Ausarbeitung einer "`Mini-VWA"' im Rahmen des VWA-Wahlpflichtfaches als Vorbereitung & Erste Beschäftigung mit einem der Teilgebiete (Multilayer Perceptron-Netzwerke) \\\hline
		Winter 2017 & Ausarbeitung der Einreichung & Einreichung ausgearbeitet \\\hline
		2018-01-28 & 2. E-Mail an die betreuende Lehrperson & \\\hline
		2018-02-15 & Einreichung des VWA-Themas & VWA-Thema eingereicht und von Betreuer und Direktorin genehmigt \\\hline
		2018-04-09 &  & VWA-Thema durch LSI genehmigt\\\hline
		2018-06-06 & 2. Besprechung mit dem Betreuer über diverse formale Kriterien wie Zitierung bei Programmcode, Aufbau der Experimente & \\\hline
		2018-06-24 \newline bis \newline 2018-07-03 & Suche nach Literatur; Herunterladen der Internetquellen um Serverausfällen und Veränderungen vorzubeugen & Fortschritt: Vorerst fertige Literatursammlung \newline Nächste Schritte: Konkretisieren der Inhaltsangabe, Lesen der Literatur \\\hline
		2018-07-04 & Ausarbeiten einer genauen Übersicht über den Inhalt der VWA & Fortschritt: Übersicht darüber, worauf genau bei Lesen der Literatur geachtet werden soll \\\hline
		2018-07-06 \newline bis \newline 2018-07-18& Beginn Intensives Lesen der Literatur mit Erstellung von Randbemerkungen & Fortschritt: Wissen, wo genau was steht, um beim Schreiben besser arbeiten zu können \\\hline
		2018-07-14 \newline bis \newline 2018-07-27& Digitalisieren der während dem Lesen gemachten Notizen für Kapitel 1 um diese während des Schreibens schneller durchsuchen zu können und das Formulieren zu üben & Fortschritt: Stoff zum Schreiben von Kapitel 1 aufbereitet \\\hline
		2018-07-13 & 2. E-Mail an die betreuende Lehrperson mit Fragen bezüglich der Schreibweise von Begriffen und der Handhabung von Programmcode in der VWA & \\\hline
		2018-07-16 & Formatierung von Programmcode und Notationstabelle in \LaTeX & \\\hline
		2018-07-18 & Beginn mit Schreiben des 2. Kapitels & \\\hline
		2018-07-28 & Fertigstellung einer ersten Version des 2. Kapitel & \\\hline
		2018-07-29 & 3. E-Mail an betreuende Lehrperson mit Fragen und dem 2. Kapitel im Anhang & \\\hline
		2018-08-01 \newline bis \newline 2018-08-02 & Schreiben des 3. Kapitels & \\\hline	
		2018-08-17 & Beginn mit Schreiben des 6. Kapitels & \\\hline
		2018-08-25 & Fertigstellung einer ersten Version des 6. Kapitels & \\\hline	
		2018-08-26 & Beginn mit Schreiben des 4. Kapitels & \\\hline
		2018-08-29 & Fertigstellung einer ersten Version des 4. Kapitels & \\\hline
		2018-08-30 & Beginn mit Schreiben des 5. Kapitels & \\\hline
		2018-09-03 & 3. Besprechung der bisherigen VWA  mit Betreuer. & \\\hline
		2018-08-29 & Fertigstellung einer ersten Version des 5. Kapitels & \\\hline
		2018-09-08 & Beginn mit Arbeit an Programmierarbeiten & \\\hline
		2018-10-03 & 4. Besprechung der bisherigen VWA, Versuchsergebnisse und dem weiteren Vorgehen & \\\hline
		2018-10-25 & Abschluss der Programmierarbeiten und Versuche & \\\hline 
		2018-11-23 & 5. Besprechung über Ergebnisse der Programmierarbeiten & Grundsteinlegung für 7. Kapitel \\\hline
		2018-12-19 & Besprechung der mathematischen Ausdrücke mit Mag.$^{\textrm{a}}$ Gertrud Aumayr & \\\hline
		2018-12-24 \newline bis \newline 2018-12-26 & Fertigstellung der Notationstabelle & \\\hline 
		2018-12-26 \newline bis \newline 2018-12-29 & Fertigstellung des Glossars & \\\hline
		2018-12-28 & Niederschreiben der Beweise in Anhang B & \\\hline
		2018-12-28 \newline bis \newline 2018-12-29 & Schreiben der Einleitung & \\\hline
		2018-12-31 \newline bis \newline 2019-01-05 & Schreiben des 7. Kapitels & \\\hline
		2019-01-19 \newline bis \newline 2019-01-20 & Schreiben des Resümees, Abstracts und Vorworts & \\\hline
		2019-01-21 \newline bis \newline 2019-02-02 & Überarbeiten der Arbeit & \\\hline
		2019-02-02 & Druck der Arbeit & \\\hline
		2019-02-11 & Abgabe der Arbeit & \\\hline
		%Besprechungen mit der betreuenden Lehrperson, Fortschritte, offene Fragen, Probleme, nächste Schritte \\
	\end{longtable}
%	\vspace*{0.4cm}
	Die Arbeit hat eine Länge von 59869 Zeichen.
	\\ \
	\\ \
	\\ \
	\\ \
	\\ \
	\\ \
	\\ \
	\\ \
	\\ \
	\parbox{6cm}{\centering\hrule
		\strut \centering\footnotesize Ort, Datum} \hfill\parbox{6cm}{\hrule
		\strut \centering\footnotesize Unterschrift}

\end{document}